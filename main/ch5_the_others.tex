\chapter{The others (\textit{not the movie})}
\section{Useful stuff for citation}
\par
This is to cite stuff in-line \verb*|cite{bla-bla}| $\xrightarrow{}$ \cite{REF:3, REF:1}, but to put citations in brackets should appear something like dis \verb*|autocite{bla-bla}| $\xrightarrow{}$ \autocite{REF:3, REF:1}.
\par
Now for brief foot citations we do it like dis by using biblatex package \verb*|footcite{bla-bla}| $\xrightarrow{}$ look down\footcite{REF:1}; for fully-detailed citation we do it like dis \verb*|footfullcite{bla-bla}| $\xrightarrow{}$ look down\footfullcite{REF:1} and another one for fun\footfullcite{REF:3}.
\par
To cite the author only we do dis \verb*|citeauthor{bla-bla}| $\xrightarrow{}$ \citeauthor{REF:1}; the only year god only knows why we could use it but here we go \verb*|citeyear{bla-bla}| $\xrightarrow{}$ \citeyear{REF:1}.
\par
This is just a normal footnote \footnote{I'm just a normal footnote minding my own business}.

\section{Useful stuff chemical formulae}
\begin{itemize}
    \item The first style:\newline~\newline
\ce{Na2SO4 ->[H2O] Na+ + SO4^2-}
\ce{(2Na+,SO4^2- ) + (Ba^2+, 2Cl- ) -> BaSO4 v + 2NaCl}
~\newline~\newline
\item The second style:\newline~\newline
\begin{chemmath}
  Na_{2}SO_{4}
  \reactrarrow{0pt}{1.5cm}{\ChemForm{H_2O}}{}
  Na^{+} + SO_{4}^{2-}
\end{chemmath}
\begin{chemmath}
  (2 Na^{+},SO_{4}^{2-}) + (Ba^{2+},2 Cl^{-})
  \reactrarrow{0pt}{1cm}{}{}
  BaSO_{4} + 2 NaCl
\end{chemmath}
~\newline~\newline
\item The third style:\newline~\newline
\schemestart
\chemfig{Na_2SO_4}
\arrow{->[\footnotesize\chemfig{H_2O}]}
\chemfig{Na^+}\+\chemfig{SO_2^{-}}
\schemestop
\schemestart
(2\chemfig{Na^+}, \chemfig{SO_4^{2-}})
\+
(\chemfig{Ba^{2+}}, 2\chemfig{Cl^{-}})
\arrow(.mid east--.mid west)
\chemname[1pt]{\chemfig{BaSO_4}}{\chemfig{-[,0.75]-[5,.3,,,-stealth]}}\+2\chemfig{NaCl}
\schemestop

\end{itemize}

\newpage
just an empty page \dots