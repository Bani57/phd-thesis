%\begingroup
%\let\cleardoublepage\clearpage


% English abstract
\cleardoublepage
\chapter*{Abstract}
~\newline~\newline~
%\markboth{Abstract}{Abstract}
\addcontentsline{toc}{chapter}{Abstract} % adds an entry to the table of contents
% put your text here
This thesis will demonstrate how graph machine learning methods can be scaled by combining holistic and reductionist perspectives. Across the different domains of knowledge graph reasoning and generative graph modeling, we will introduce a series of techniques that balance abstraction and fine-grained detail. Our COINs framework provides a principled approach to accelerating link prediction and query answering through community-based coarsening, supported by both theoretical guarantees and industrial validation. For generative tasks, we developed models showing how scalable graph synthesis can be achieved without sacrificing structural fidelity. Finally, we will explore the synergy between reasoning and generation by applying diffusion processes to anomaly correction in knowledge graphs, illustrating that edge-centric and distributional modeling can converge toward complementary solutions. Altogether, the work will underscore that scalable graph learning is best achieved not by choosing between holism and reductionism, but by weaving them together into a unified methodology.

\vskip0.5cm
Keywords: Community Detection, Knowledge Graph Embeddings, Generative Graph Models, Diffusion Sampling, Scalable Inference
%put your text here


% % German abstract
% \begin{otherlanguage}{german}
% \cleardoublepage
% \chapter*{Zusammenfassung}
% ~\newline~\newline~
% %\markboth{Zusammenfassung}{Zusammenfassung}
% % put your text here
% \lipsum[1-2]
% \vskip0.5cm
% Stichwörter: 
% %put your text here
% \end{otherlanguage}




% % French abstract
% \begin{otherlanguage}{french}
% \cleardoublepage
% \chapter*{Résumé}
% ~\newline~\newline~
% %\markboth{Résumé}{Résumé}
% % put your text here
% \lipsum[1-2]
% \vskip0.5cm
% Mots clefs: 
% %put your text here
% \end{otherlanguage}


%\endgroup			
%\vfill
