\chapter*{Acknowledgements}
~\newline~\newline~
\markboth{Acknowledgements}{Acknowledgements}
\addcontentsline{toc}{chapter}{Acknowledgements}
% put your text here
%\lipsum[1-2]

For a person researching models for relational data, ironically, I sure do often have difficulty making new connections with real people. Nevertheless, the support from my local network that I have received during these last 6 years of postgraduate studies has been enormous, and without it, I would have been working for nothing.
%\\

First, I would like to sincerely thank my advisor, Prof. Volkan Cevher, for accepting to supervise my master's thesis all those years ago, investing in the belief that state-of-the-art research could emerge from those humble beginnings, and inviting me as a fresh lion cub to the pride. Speaking of which, my day-to-day would not be the same without my mates from the noisy office: Luca, Pedro, Elias, Pol, Ioannis, Arshia, Ali, Stratis, Grigoris, and God will not forgive me if I do not include here also my ever-charming left-side neighbor, Kimon. The Mediterranean Sea is very beautiful, no doubt about it, but bringing so many of its residents into a confined space surely makes life special. But let's not also forget to thank the \enquote{quiet savants} of the lab for their sincere, friendly attitude: Thomas S., Paul, Fabian, Leello, Thomas P., Fanghui, Yongtao, Zhenyu, Wangyun, Frank, and Leyla. Everything about talkativeness aside, watching my colleagues' accomplishments and collaborating with them has always reminded me how glad I am to be a part of such a world-renowned scientific and academic institution. Never forget the moments we shared together. I would single out the two remaining LIONS lab members for their specific contributions. The same as basically all previous thesis works originating from our lab, this text cannot understate the importance of our admin, Gosia. While the rest of us do our \enquote{research}, she fights for our tight group against the world on the battlefield of Swiss bureaucracy and conference websites. Also, being the only two Slavic people supporting different Premier League football teams has allowed us to have a unique mutual understanding. On another note, hats off to my amazing supervisor, co-author, and fellow graph enthusiast, Igor. I have never seen such academic and personal integrity elsewhere. I am an only child, but if there was someone who I would have to put in the older brother role, I would not have dilemmas. Last but not least, many thanks to our semester project students Xiyun, Brando, Niels, and Povl for their immense help with our graph machine learning research.
%\\

General salutations as well to the country of Switzerland, the canton of Vaud, the happy city of Lausanne on the hills, and finally, l'École Polytechnique Fédérale de Lausanne for welcoming my application to complete my postgraduate studies here. Society here is truly almost flawless, though something does need to be settled with the Alps' hate of warm air and the month of November in general. On the other hand, I would never have even arrived here without my fatherland, Macedonia, providing me with the essential knowledge a machine learning researcher needs, as well as financial support that made my hopes of a master's degree abroad realistic. I would like to thank my bachelor's thesis mentor, Prof. Sonja Gievska from the Faculty of Computer Science and Engineering at the Ss. Cyril and Methodius University in Skopje, who supported my academic inspirations as a fellow scientist educated abroad, and provided me with the first experience in paper writing and academic publishing.
%\\

Next, compared to most PhD students, I have been extremely lucky to be able to have a third category of acknowledgements to add! This thesis work was partially supported by Swisscom (Switzerland) AG, the lead telecom service provider in Switzerland. It is obvious that ever since I received my internship project from them about scalable knowledge graph reasoning in the last master's semester, I have become entranced by the topic. I will never forget my time collaborating with my colleagues from the Digital Lab: Emma, Natalie, Vincent, Claudiu, Dan, as well as Sam, Milena, and Nicola from the Monty team. I would single out Vincent, who, funnily enough, even though a year younger, started as my manager, then became my co-author, and finally someone I believe has become one of the few close friends I have gained in my adult years. Last but not least, shout-outs to my and Swisscom's master's students, Arnaud and Catalin, with whom we spent time discussing knowledge graph abstractions. Their displayed academic excellence resulted in a rare need for my supervision. 
%\\

While I am in the context of saluting indispensable enterprises, let us bow down to the king of retail, Lidl Lausanne-Flon, for single-handedly sustaining my needs these past years without changing those affordable prices that are so rare in Switzerland, as well as Discord and Viber, for providing a cost-free manner to communicate non-stop with loved ones, continuing to share happy moments together despite being separated by more than 1500 km. On that note, there's nothing more I could recommend in life than not losing contact with childhood friends, no matter where your coordinates are currently. That is the true Macedonian spirit. I was blessed to be able to attend the weddings of people I had met at 7 years old and shared everything else since. Apologies for my backseat gaming, zero football skills, and scientific righteousness to: Andrej, Atanas, Kalin, Goran, Ivan, Dorian, Viktor, and many others.
%\\

To conclude, ohana does not mean family; it means everything. When nobody else remains, a relative is always there to lend a hand. One big 
\selectlanguage{macedonian}
„БЛАГОДАРАМ“
\selectlanguage{english} 
to my parents, uncles, aunts, and grandparents for raising me with proper values and providing the emotional support necessary for me to finish my education and build a successful career. Here I would put on the pedestal my mother, Aleksandra, who is my pillar of strength, as I could never have the optimism about life that she always does. Together we share not just our honest thoughts, but the hobby of travelling as well. Indeed, my happiest memories come from seeing the world just the two of us, and there are still the continents of South America and Australia left to explore (let's leave the penguins alone). She does also hate it when I attempt to find ways to pay her back for the immense monthly financial support she provided during my 2 years of master's, but I will continue, same as how she also continued to visit me in Switzerland frequently at periods of lonely feelings during my PhD. 

\bigskip

\begin{center}
    \emph{{\LARGE Scale up your love to new heights.}}
\end{center}

\bigskip
 
\noindent\textit{Lausanne, \today}
\hfill A.~J.
