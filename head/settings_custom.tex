%%%%%%%%%%%%%%%%%%%%%%%%%%%%%%%%%%%%%%%%%%%%%%
%
%		Thesis Settings
%		Custom settings
%
%		October -- 2019
%
%%%%%%%%%%%%%%%%%%%%%%%%%%%%%%%%%%%%%%%%%%%%%%

%
%   Use this file for your own custom packages, command-definitions, etc...
%


% the following lines are for creating a simplified TO-DO box. However since boites is not per default installed with all latex-distributions, we have removed this example again
% if you want to use it and do not have "boites" installed, you can get it from here: http://www.ctan.org/tex-archive/macros/latex/contrib/boites
%
%{boites,boites_exemples}
%\newcommand{\todolist}[1]{\begin{boiteepaisseavecuntitre}{TO DO in this chapter} #1 \end{boiteepaisseavecuntitre}}  % creates a little box
% %\newcommand{\todolist}[1]{}  % to be used when to do is not to be printed

% *********** Change the header style ***********
\pagestyle{fancy}
\renewcommand{\chaptermark}[1]{\markboth{#1}{#1}}
\fancyhead[RE]{\textbf{\leftmark}}
\fancyhead[LE]{\textbf{\chaptername\ \thechapter\ }}
\fancyhead[LO]{\textbf{\leftmark}}
\fancyhead[RO]{\textbf{\chaptername\ \thechapter\ }}

% if you use chapter* instead of chapter use %\chaptermark{Introduction} o explicitly .. 
% to do some oher fancy stuff click this link:
%https://en.wikibooks.org/wiki/LaTeX/Customizing_Page_Headers_and_Footers#Style_customization

% *********** Chemistry Libraries ***********
\usepackage[version=4]{mhchem}
\usepackage{chemist}
\usepackage{chemfig}

% *********** The APA citation style ***********
\usepackage{csquotes} % needed for biblatex
\usepackage[backend=bibtex,style=ieee]{biblatex} % uses APA style
\addbibresource{tail/Thesis.bib}

% Additional library to copy and paste the references in-line (not used here)
% \usepackage{bibentry}
% \addbibresource{tail/References.bib}
% \nobibliography*
% \bibliography{tail/References.bib}

% *********** Adjusting the footnotes symbols ***********
% Uncomment only the needed one here XD:

% Option [1]: To only use roman enumeration
% \renewcommand{\thefootnote}{\Roman{footnote}}

% Option [2]: To use Arabic numbering (The default option)
\renewcommand{\thefootnote}{\arabic{footnote}} % 

% Option [3]: To symbols instead of numbers (un comment both lines)
% \usepackage[symbol]{footmisc}
% \renewcommand{\thefootnote}{\fnsymbol{footnote}}

% Option [4]: To use different styles for normal notes and citations' notes:
% The source: https://tex.stackexchange.com/questions/408527/differentiate-between-footcite-and-footnote
% \usepackage{manyfoot}
% \newfootnote{A}
% \newcounter{footnoteA}
% \newcommand{\footnoteA}{%
% \stepcounter{footnoteA}%
% \Footnotemark\thefootnoteA \FootnotetextA{}}
% \renewcommand{\thefootnoteA}{\alph{footnoteA}}

% *********** The end is near ***********

\definecolor{darkred}{rgb}{0.75, 0.0, 0.0}
\definecolor{crimson}{rgb}{0.86, 0.08, 0.24}

% If you use the hyperref package, please uncomment the following two lines
% to display URLs in blue roman font according to Springer's eBook style:
\usepackage[breaklinks]{hyperref}
\hypersetup{pdfborder={0 0 0},
	colorlinks=true,
    citecolor=darkred,
    linkcolor=darkred,
    urlcolor=crimson}
\urlstyle{same}
\PassOptionsToPackage{hyphens}{url}
% \urlstyle{rm}

\usepackage{booktabs}						% professional-quality tables
\usepackage{multirow}						% tabular cells spanning multiple rows
\usepackage{amsfonts}						% blackboard math symbols

%%% NEW PACKAGES
\usepackage{cleveref}
\usepackage{amsmath}
\usepackage{amssymb}
\usepackage{mathtools}
\usepackage{amsthm}
\usepackage{enumerate}   
\usepackage{longtable}
\usepackage{adjustbox}
\allowdisplaybreaks
\usepackage{algorithm}
\usepackage{algorithmic}
\newlength\myindent
\setlength\myindent{2em}
\newcommand\bindent{%
  \begingroup
  \setlength{\itemindent}{\myindent}
  \addtolength{\algorithmicindent}{\myindent}
}
\newcommand\eindent{\endgroup}\usepackage{wrapfig}
\raggedbottom

\DeclareMathOperator*{\argmax}{arg\,max}
\DeclareMathOperator*{\argmin}{arg\,min}
\theoremstyle{definition}
\newtheorem{definition}{Definition}
\theoremstyle{plain}
\newtheorem{proposition}{Proposition}
\theoremstyle{plain}
\newtheorem{lemma}{Lemma}
\theoremstyle{plain}
\newtheorem{theorem}{Theorem}

\numberwithin{lemma}{section}
\numberwithin{proposition}{section}
\numberwithin{definition}{section}
\numberwithin{theorem}{section}
\numberwithin{figure}{section}
\numberwithin{table}{section}
\numberwithin{algorithm}{section}

\relpenalty=10000
\binoppenalty=10000

% \usepackage{nath}

\newcommand{\of}[1]{\mathopen{}\left(#1\right)}
\newcommand{\off}[1]{\mathopen{}\left[#1\right]}
\newcommand{\offf}[1]{\mathopen{}\left\{#1\right\}}
\newcommand{\ang}[1]{\mathopen{}\left\langle#1\right\rangle}
\newcommand{\abs}[1]{\mathopen{}\left|#1\right|}

\usepackage{tikz}
\usepackage{environ}
\makeatletter
\newsavebox{\measure@tikzpicture}
\NewEnviron{scaletikzpicturetowidth}[1]{%
  \def\tikz@width{#1}%
  \def\tikzscale{1}\begin{lrbox}{\measure@tikzpicture}%
  \BODY
  \end{lrbox}%
  \pgfmathparse{#1/\wd\measure@tikzpicture}%
  \edef\tikzscale{\pgfmathresult}%
  \BODY
}
\makeatother